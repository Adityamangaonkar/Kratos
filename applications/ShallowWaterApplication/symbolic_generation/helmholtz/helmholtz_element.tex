\documentclass[a4paper,12pt]{article}

\usepackage[utf8]{inputenc}
\usepackage[english]{babel}
\usepackage{amsmath}
\usepackage{amsfonts}
\usepackage{amssymb}
\usepackage{bookmark}

\title{Standard Helmholtz wave equation element symbolic implementation}
\author{Miguel Masó Sotomayor}
\date{\today}

\begin{document}
\maketitle



\section{Standard Helmholtz wave equation formulation}

Shallow water equations derive from Navier-Stokes quations, assuming fully incompressibility, the vertical velocity $u_{3}$ to be small and the corresponding acceleration negligible. We now have the momentum conservation:

\begin{equation}
\frac{\partial\mathbf{u}}{\partial t} - 
\frac{1}{\rho}\nabla\cdot\mathbf{\sigma} + 
\mathbf{a}\cdot\nabla\mathbf{u} = 
\mathbf{f}
\end{equation}

and the mass conservation:

\begin{equation}
\nabla\cdot\mathbf{u} = 0
\end{equation}

Vertical coordinate inegration...

In omission of all non-linear terms, bottom drag, etc., and approximately $h~H$, and substituting $H\mathbf{U}$ term, shallow water equations can be written as the standard Helmholtz wave equation

\begin{equation}
\frac{\partial^{2}\eta}{\partial t^{2}} - 
\nabla\cdot\left(
	gH\mathbf{I}\cdot\nabla\eta
\right) = 0
\end{equation} 

Once arrived at this point, the Helmholtz equation residual can be defined as

\begin{equation}
R\left(\eta	\right) = 
\frac{\partial^{2}\eta}{\partial t^{2}} - 
\nabla\cdot\left(
	gH\mathbf{I}\cdot\nabla\eta
\right)
\end{equation}



\section{Galerkin weak form}

Considering the free surface elevation test function $\upsilon$, the
Galerkin weak form of the problem can be obtained as

\begin{equation}
\int_{\Omega} \upsilon R\left(\eta\right) d\Omega = 0
\end{equation}

Developing terms int the residual Galerkin residual weak form one obtains

\begin{equation}
\int_{\Omega^{e}} \upsilon \frac{\partial^{2}\eta}{\partial t^{2}} d\Omega + 
\int	_{\Omega^{e}} \nabla \cdot \upsilon gH\mathbf{I} \cdot \nabla\eta d\Omega -
\oint_{\Gamma^{e}} \upsilon\hat{q} d\Gamma = 0
\end{equation}



\subsection*{Temporal approximation}

Using a backward difference scheme

\begin{equation}
%\frac{\partial^ {2}\eta_{n}}{\partial t^ {2}} \approx
\ddot{\eta}_{n} =
\frac{\eta_{n} - 2\eta_{n-1} + \eta_{n-2}}{\Delta t}
\end{equation}


\section{Symbols to be employed}



\section{Implementation}



\end{document}
